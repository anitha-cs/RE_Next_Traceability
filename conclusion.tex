\section{Conclusion}




%\section{Related Work}
%
%Establishing traceability, especially in MDB is not a new topic, but has
%
%One of the major challenges in MDB is precisely establishing traceability between the system requirements and the parts of the model that implements it. The main reason behind the difficulty is the inherent complexity of the models that makes the process of manually establishing trace links exhausting. However, there has been some research efforts to automatically establish the traceability between requirements and model elements. Some of them include :
%\begin{itemize}
%  \item Event-Based Traceability (EBT) (Cleland-Huang et al) is a method that automatically establishes trace links and maintains it, between requirements to models, using event-based mechanisms.
%  \item Egyed and Grunbacher enhance the basis from which they derive links by following the approach of dynamic program analysis: They record program execution traces (these are—as described earlier—dynamic program behavior logs) from the execution of test-cases. They combine the results with a set of requirements-to-code traceability links, which have to be established manually beforehand and infer traceability links between requirements.
%  \item Wenzel et al. describe a tool which is able to compare the version history of models, to derive traceability links, and consequently, to reconstruct the evolution steps of single model elements.
%  \item alloy work here or at the end?
%\end{itemize}
%\anitha{I have a survey paper, I will fill in relevant work here}
%
%While the above techniques establish trace links, without a rationale for why and how they were established they are often inadequate to perform useful analysis in practice~\cite{Trace links explained: An automated approach for generating rationales}. In that context, \textbf{\emph{Satisfaction Arguments}} offer a meaningful way to establish traceability between functional requirements and its realization. Originally proposed by Zave and Jackson, satisfaction argument is an approach to demonstrate how the behaviours of the system along with the assumptions made about its environment satisfy the functional requirements. These arguments have been enormously useful in establishing traceability between the requirements and parts of the system that implements them. Unlike traditional traceability that just links two artifacts, satisfaction argument is aimed to provide an explanation to why the traceability was established.
