\section{Discussion}

While the notion of correctness and completeness in requirements traceability is not a new topic of discussion, our argument is that it is not been rigorously defined and assessed. The intent of this RE@Next! paper is to clarify the foundations of tracing requirements to its realization - in particular the requirements satisfaction traceability -- and explaining its the benefits and ramifications.

While there are enormous amount work in the area of requirements traceability, there are no, to the best of our knowledge, rigours yardsticks to quantitatively and qualitatively access the trace links. This raises a qualm about the way traceability is assessed. Hence, we believe that there is a strong need to identify metrics, define approaches and develop tools that will rigorously assess trace links.

In this paper we define a strong theoretical framework to understand and assess requirements satisfaction traceability, that we believe is an initial attempt towards addressing the above need. By defining a requirements satisfaction trace from a requirement to a set of support, rather than a target artifact, we intertwine the rationale to verify/validate its correctness automatically. Such traces are self-sufficient to be verified for correctness, without the need for another source of truth or spending enormous amount of time to validate them. By examining these traces one could verify and validate if the system indeed satisfies the requirements in s intended ways. Further, this helps redefine the notion of completeness in traceability from capturing trace links to all requirements to capturing *all* trace links to all requirements. It is worth reiterating that our goal is not yet another formalization of traceability but to provide the foundation for any future research on satisfaction trace link generation and assessment.

Contrary to the general notion that it is impossible or extraordinarily difficult to identify all trace link in practice~\cite{stravsunskas2002traceability}, some of the initial results from our recent efforts~\cite{2016arXiv160304276G} indicate that such complete requirements satisfaction trace can be established, in fact automatically in the realm of formal methods and model based developments. 
Previously~\cite{hilt2013}, we demonstrated a model based approach to system construction in which compositional proofs are used to automatically establish satisfaction arguments. The approach was based on mathematically proving the requirements with respect to a model of the system. As an extension of that, we recently developed a technique~\cite{2016arXiv160304276G} to extract the elements of the proof, that we call {\em inductive validity cores (IVC)}, to present an explanation to how the requirements were verified in the model. These IVCs are nothing but the set of support for a requirement. Our next step is to use this approach iteratively to extract all the sets of support for all the requirements and establish a complete traceability for the system. While the details the approach and initial results are not in scope of this paper, an interested reader is directed to~\cite{2016arXiv160304276G}.

