\section{Introduction}

\emph{Requirements traceability} -- ``the ability to describe and follow the life of a requirement, in both a forwards and backwards direction (i.e., from its origins, through its development and specification, to its subsequent deployment and use, and through all periods of on-going refinement and iteration in any of these phases)."~\cite{gotel} -- has been a topic of great interest in research and practice for several decades. Intuitively, it concerns establishing relationships, called \emph{trace links}, between the requirements and one or more artifacts of the system. Among the several different development artifacts and the relationships that can established from/to the requirements, being able to establish trace links from requirements to artifacts that realize or \emph{satisfy} those requirements -- particularly to entities within those artifacts, that we will call \emph{target artifacts} in the rest of the paper -- has been enormously useful in practice. For instance, it helps analyze the impact of changes in one artifact on the other, assess the quality of the system, aid in creating assurance arguments for the system, etc. In this paper, we focus our attention to this subset of requirement traceability, that we call \emph{Requirements Satisfaction Traceability.}

Instead of just recording the trace links from each requirement to the target artifacts, \emph{Satisfaction Arguments}~\cite{zave1997four} offer a semantically rich way to establish them. Originally proposed by Zave and Jackson, a satisfaction argument demonstrates how the behaviors of the system and its environment together satisfy the requirements. From a traceability perspective, these arguments help establish trace links (the \emph{satisfied by} relationship) between the requirements and those parts of the system and environment (the target artifacts) that were necessary to satisfy the requirements; We call those target artifacts the \emph{set of support} for that requirement. Mathematically, if we think of the argument as a proof in which the requirement is the claim, then the set of support are the axioms (or clauses) that were necessary to prove the claim and the trace links are means to associate the claim to those clauses. Such proofs are not, in general, unique, and often there are multiple sets of clauses that could be used to construct the proof.  Neither Zave and Jackson nor (in our examination) the existing traceability literature discusses multiple alternative satisfaction arguments or sets of trace links for one system design.

%This casts a doubt on how traceability is perceived in practice.  Mike: why?  This seems a silly thing to say.  In practice people don't doubt traceability because they don't know this problem exists.

While it is standard practice to refer to describe a satisfaction argument as capturing the relationship between artifacts, in our opinion, at least when discussing requirements satisfaction traceability, it is beneficial to distinguish between trace links to ``a'' set of support vs. ``the" sets of support. In so doing, it encourages a substantial change in understanding what traceability {\em means} and how traceability be used. For instance, in a safety critical system, one could intentionally add multiple ways to satisfy a requirement for fault tolerance or one could unintentionally add system functionality that leads to multiple ways of satisfying a requirement. Although for proof of satisfaction, one set of support is sufficient, it is insufficient for other interesting analyses that can be performed using trace information. For example, when there is a change in an implementation element required by the argument, impact analysis may be overly conservative; alternate paths to satisfy the requirement may be unaffected.  In addition, when comparing multiple approaches toward constructing satisfaction arguments (e.g., manual approaches vs. automated or semi-automated approaches), it is possible that the approaches are constructing different, but equally valid arguments.

%automated approaches for determining satisfaction argument trace links
%the lack of knowledge of all sets of support might result in either misplaced confidence in the system since the old requirement might still be satisfied by alternate means or spend enormous amount of time to identify all the alternate satisfactions and re-engineer them. \anitha{I need to rethink if this example emphasises the point.}

Establishing trace links to all sets of support, that we call \emph{complete} traceability, provides us insight about the elements of the set of support - ones that are necessary, the ones that are optional and the ones that do not contribute to satisfying a requirement.  We categorize the elements as \emph{Must}, \emph{May} and \emph{Irrelevant} support elements for each requirement.  This categorization of support elements is useful in several ways, such as (a) to precisely assess the influence of change of one artifact on the other, called impact analysis, (b) to understand if all requirements and how they are satisfied in the system (c) to identify which parts of the system are not necessary to satisfy any requirements, so called ``Gold Plating", (d) to locate parts of the system that are most critical in terms of satisfying most requirements, (e) to identify system assumptions that are used by most requirements, and so on. In fact, establishing complete traceability helps identify unintended ways the requirement was satisfied by the system and its environment.

In this paper, we examine how this notion of completeness in traceability changes the way it is perceived, established, maintained, and used. We introduce and discuss the notion of completeness in traceability, which considers *all* \emph{satisfied by} trace links between the requirements and the target artifacts that work to satisfy the requirements, and contrast it with the partial traceability common in practice.  The intent of this RE@Next! paper is to highlight the aspect of completeness in traceability and discuss the ramifications of incomplete (or partial) traceability. %We begin our discussion with a short motivating example that illustrates the notion of multiple sets of support for a requirement. Following the example, in Section~\ref{}we provide a formal definition to a trace, set of support and complete traceability
%\anitha{Since its a short paper, do we need a overview of paper? We do have space. }
%\mike{Do we want to go here?  I'm leaving it in, but I'm not sure that we need or want this.}  While establishing complete traceability may seem extraordinarily difficult to establish in practice~\cite{stravsunskas2002traceability}, our hypothesis is that, given use of formal methods and model based development, it can be achieved in an automated and efficient manner.  We have recently investigated {\em inductive validity cores}~\cite{2016arXiv160304276G} as a mechanism to produce sets of support and believe it can be generalized towards all cores.
