\section{Introduction}

\emph{Requirements traceability} -- the ability to describe and follow the life of a requirement in system development~\cite{gotel} -- has been a topic of great interest in research and practice for several decades. Intuitively, it concerns establishing relationships, called \emph{trace links}, between the requirements and one or more artifacts of the system. Among the several different development artifacts and the relationships that can established from/to the requirements, being able to establish trace links from requirements to artifacts that realize or \emph{satisfy} those requirements -- particularly to entities within those artifacts, that we will call \emph{target artifacts} in the rest of the paper -- has been enormously useful in practice. For instance, it helps analyze the impact of changes in one artifact on the other, assess the quality of the system, aid in creating assurance arguments for the system, etc. In this paper, we focus our attention to this subset of requirement traceability, that we call \emph{Requirements Satisfaction Traceability.}

Instead of just recording the trace links from each requirement to the target artifacts, \emph{Satisfaction Arguments}~\cite{zave1997four} offer a semantically rich way to establish them. Originally proposed by Zave and Jackson, satisfaction argument is an approach to demonstrate how the behaviors of the system and its environment together satisfy the requirements. While one could trivially establish this argument between the entire system-environment and the requirement, the intent of the satisfaction argument is to capture only those parts of the system and environment that are necessary to satisfy the requirement. From a traceability perspective, these arguments help establish trace links (the \emph{satisfied by} relationship) between the requirements and those parts of the system and environment (the target artifacts) that were necessary to satisfy the requirements; We call those target artifacts the \emph{set of support} for that requirement. Mathematically, if we think of the argument as a proof in which the requirement is the claim, then the set of support are the axioms (or clauses) that were necessary to prove the claim and the trace links are means to associate the claim to those clauses. However, such proofs are not, in general, unique, and often there are multiple sets of clauses that could be used to construct the proof.  Neither Zave and Jackson nor (in our examination) the existing traceability literature discusses multiple alternative satisfaction arguments or sets of trace links for one system design. This casts a doubt on how traceability is perceived in practice.

While it is has been a general practice to refer to traceability as capturing the relationship between artifacts, in our opinion, atleast when discussing requirements satisfaction traceability it is beneficial to distinguish between trace links to ``a'' set of support vs. ``the" sets of support. In so doing, it allows a substantial change in understanding what traceability {\em means} and how traceability be used. For instance, in safety critical system domain, for fault tolerance one could intentionally add multiple ways to satisfy a requirement or one could unintentionally add system functionality that leads to multiple ways of satisfying a requirement. Although, to argue or prove that the system satisfies its requirement it might be suffice to get one set of support, it is insufficient for other analysis performed using trace information. For example, when there is a change in a requirement, the lack of knowledge of all sets of support might result in either misplaced confidence in the system since the old requirement might still be satisfied by alternate means or spend enormous amount of time to identify all the alternate satisfactions and re-engineer them. \anitha{I need to rethink if this example emphasises the point.}

Establishing trace links to all sets of support, that we call \emph{complete} traceability, provides us insightful information about the elements of the set of support - ones that are absolutely necessary, the ones that are optional and the ones that do not contribute to satisfying a requirement that we categorize them as \emph{Must}, \emph{May} and \emph{Irrelevant} support elements for each requirements. This categorization of support elements is useful in several ways, such as (a) to precisely perform impact analysis, (b) to analyse if changing a part of the system affects the way requirements are satisfied, (c) to identify which parts of the system does not contribute to satisfying any requirement of the system, the so called ``Gold Plating", (d) to locate parts of the system that are most critical in terms of satisfying most requirements, (e) to point to system assumptions that are used by most requirements, and the list goes on. In fact, it helps identify the unintended ways the requirement was satisfied by the system and its environment.

In this paper, we examine how this notion of completeness in traceability changes the way it is perceived, established, maintained, and used. We introduce and discuss the notion of completeness in traceability, which considers *all* \emph{satisfied by} trace links between the requirements and the target artifacts that work to satisfy the requirements, and contrast it with the partial traceability common in practice.  The intent of this RE@Next! paper is to highlight the aspect of completeness in traceability and discuss the ramifications of incomplete (or partial) traceability. While establishing complete traceability has been considered impossible or extraordinarily difficult to establish in practice~\cite{stravsunskas2002traceability}, our our hypothesis is that, in the realm of formal methods and model based developments, it can be achieved in an automated and efficient manner. Our hypotheses is based on the initial results we got from our recent work on automatically extracting the set of support when formally verifying the requirements of systems using a model based approach.
