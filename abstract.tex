When establishing associations, known as \emph{ trace links}, between a requirement
and the artifacts that lead to its satisfaction, it is essential to know what the links mean. While research into this type of
traceability---what we call \emph{Requirements Satisfaction Traceability}---has been an active research area for some time, none of the
literature discusses the fact that there are often multiple ways in which a requirement can be satisfied---there are multiple satisfaction arguments.
The distinction between establishing a single satisfaction argument
between a requirement and its implementation (tracing \textbf{one} way the requirement is implemented) vs. tracing \textbf{all} satisfaction arguments, and the possible ramifications for how the trace
links can be used in analysis, has not been well studied.  We examine how this distinction
changes the way traceability is perceived, established, maintained, and used.  In this RE@Next! paper, we introduce
and discuss the notion of ``complete" traceability, which considers \textbf{all} trace
links between the requirements and the artifacts that work to satisfy the
requirements, and contrast it with the partial traceability common in
practice.  