
\begin{enumerate}

  \item Requirements traceability is the association that is established between the requirements and other development artifacts in the system such as design, code, test cases etc.

  \item Such traceability information has a number of uses. Impact analysis, Dependency analysis and satisfaction argument etc. Give example for each.

  \item Our focus in this paper is discussing requirements traceability in the context of MBD

  \item MDB is taking off in the safety critical system domain, in which models of the system are developed using tools such as Simulink, Stateflow etc. This has been a very successful development technique since it helps identify errors early in the development cycle.

  \item However, with increase in complexity of the developed models it is not easily discernible to identify how and where the requirements are in the models. In fact, to cope with complexity, it has been a common practices to decompose the task of modeling the entire sysstem into smaller and manageable parts, that are individually developed and verified and then they are composed using some technique to ensure if they are satisfied. How and where the requirements are implemented in the models is not easy to see.

  \item In this context, We believe that traceability in MBD between requirements and models is highly beneficial for various purposes...

  \item Our hypothesis is that by leveraging the capability of the tools that verify requirements, we can establish trace links between the requirements and the model. In this paper, we elaborate on our first successful attempt in validating our hypothesis.

  \item In previous work, using compositional verification techniques we verified if the system level requirements of system that is decomposed into components, is satisfied by its component level requirements given the system level assumptions. We used a compositional reasoning tool, \textbf{AGREE}.

  \item The tool produces a proof of correctness for the system level requirements, whose claims are nothing but the component level requirements and system level assumptions. By extracting the claims, we get what we call "set of support", the necessary elements for the proof. This set of support is nothing but those component requirements and assumptions that contributed to satisfying the requirement. The

  \item While one set of support is one way the model satisfies the requirement. it is interesting to know if there are more than one set of support. To our surprise Yes. Then, if we get all possible trace links, then what we get is a complete traceability between models and requirements.

  %\item the intent of this paper is not explain the implementation details of how we generated the trace links - that is scope of the other paper - but to discuss the practical implications of those results from traceability perspective

  \item By generating all the set of support, we can gain insightful information about he system. By classifying the elements in the set of support into MAY into MUST elements - we can answer a variety of interesting questions, that traceability community wishes to get answered. For example: We can know which components requirements can be changed without affecting the were

  \item While we use 3 case examples in this paper to illustrate the approach, we have elaborately tested it using more than 300 examples.

\end{enumerate}

\textbf{what is functional requirements traceability and why is it important}
\emph{Traceability} in system development in its simple sense is the association that is established between artifacts for a specific purpose. Associating the requirements of the system with other related artifacts such as design, code, test case etc., is called \emph{Requirements Traceability}. An important subset of requirements traceability involves tracing the {\em functional} requirements, that describe expected behaviors of the system to be constructed, to the actual parts of the system that realizes those requirements. Such an association is useful for various purposes such as analysing how requirements are realized in the system, identify how a change in requirements affects the existing realization, determine which parts of the system does not realize any requirement, check if the requirements sufficiently describe all aspects of the system behaviour and much more. \anitha{do i need to elaborate on the uses?}

\textbf{SA to establish FRT.}
\emph{Satisfaction Arguments} offer a meaningful way to establish traceability between functional requirements and its realization. Originally proposed by Zave and Jackson, satisfaction argument is an approach to demonstrate how the behaviors of the system along with the assumptions made about its environment satisfy the functional requirements. These arguments have been enormously useful in establishing traceability between the requirements and parts of the system that implements them. Unlike traditional traceability that just links two artifacts, satisfaction argument is aimed to provide an explanation why the traceability was established. To establish such a traceability, current techniques rely on either existing or manually created links between the artifacts such as matching texts, comments and identifiers, absence of which makes the task of establishing traceability challenging. The difficulty is exacerbated in developments such as safety critical systems that are developed using model based techniques, in which complex models of the system are developed without direct/easily inferrable links to trace to requirements. %Nevertheless, traceability is crucial in such systems.



\emph{Traceability} in system development in its simple sense is the association that is established between artifacts for a specific purpose. Establishing an association between the requirements of the system and other artifacts that are related to it in, inorder to help address a specific purpose, in other words, to answer certain questions about the relationship, is called \emph{Requirements Traceability}. Gotel et.al classified the types of traceability to requirements into two - Pre-RS traceability that referred to associations between the requirements and its sources that lead to it and Post-RS traceability that refers to the associations between the requirements and the artifacts it influenced.\textbf{ In this paper, we focus on the post-RS traceability and that is what we mean by requirements traceability in the rest of the paper.}  What we need to identify the artifacts and determine their association. To identify the artifact that we want to associate a requirement it is essential to systamtically understand how the requirements are used in the subsequent phases of development.

\anitha{System hierachical development  - requirements flow down}
Systems are naturally constrcuted in hieracjies. How system requirements are satsified by the components is crucial. Hence traceability is crucial in these developments.\textbf{ We are interested in establishing traceability between requirements at multiple levels of abstraction. }While, approaches like Refernce model for traceability helped identify which association should be established based on the artifacts involved, we strongly believe that purpose for which traceability is sought is a crucial factor to determine the association that should be established. \anitha{i need to say something about the purposes in general before I move forward}.

\anitha{Classification of purpose of traceability. I would like this to be like a table... type|artifacts|specific use}
We could classify some of the well known purposes for requirements traceability broadly into into two types, depending upon whether traceability is from or to the requirements: \anitha{classification is based on Rich traceability paper, Pohl book. there are many subcategories, that sounded repetitive}
\begin{itemize}
  \item \textbf{Impact analysis} - To determine the influence of the requirements on the associated artifact. Associations are typically established from the requirements to that artifact, called forward traceability. Impact analysis is useful for (a) change management - Will a requirement change affect the system components (b) Acceptance - Are all requirements implemented by its components? (c) Accountability - Which component is the requirement allocated to.
  \item \textbf{Dependency analysis} -  It helps determine how the associated artifact was influenced by the requirement. Question it addresses is : What requirements have given rise to the component's requirements?. This is the opposite of impact analysis. This is useful for : (a) Gold plating - Which parts of the components do not trace to any requirement? (b) Re-engineering - Which requirement will be affected if the component is changed. and (c) Quality - What is coverage of the requirements over the components?
   \item \textbf{Satisfaction analysis} -  A specialized form of dependency analysis, introduced by Jackson and Zave, is a Satisfaction Arguments. This is different from the traditional traceability between two artifacts, since it highlights the role of assumptions (the third artifact) that is necessary in addition to component artifacts to satisfy a requirement. This has been enormously useful in (a) Verification and Validation and (b) Assurance. Associations are of type : satisfies.
\end{itemize}

\anitha{I need to write about rich traceability. AND and OR}
Precisely deciding and understand the association is the tricky part. The typical associations established requirements to components are " allocated to" ; from requirements to components is "satisfied by/implemented by". What is unclear is if the association is complete? Is it the only way to associate the artifacts or are there alternate ways? If so, what are they and how do we know if we got all of them? To paritally address this concern for satisfaction arguments, \emph{Rich traceability} introduced the notion of AND and OR composition.However, this approach was based on traceable links in the artifacts, that had to be present/put in by humans that makes it time consuming, error prone.... \anitha{i dont know what to say}. We want to build a complete satisfaction argument that captures all the associations between the requirements and its components. This way it achieves far more purposes than just satisfaction.

\anitha{I want to transition to MBD nicely here... }
Among the several development techniques, we focus on MBD in which traceability is very hard to acheive yet a regulatory requirement. IN MBD, artifacts are models and formal representations of requirements. The advantage of MBD is the sophisticated tools that automatically verify the requirements with respect to the model. Our hypothesis is by leveraging the verification capability of the tools, we could establish trace links between the requirements and the model to give us a complete traceability of requirements. In this paper, we elaborate on our first successful attempt in validating our hypothesis. While we use 3 case examples in this paper to illustrate the approach, we have elaborately tested it using more than 300 examples.

For the work desribed in this paper, we choose a compositional reasoning tool, \textbf{AGREE}, that automatically verifies if the system level requirements are satisfied by its componnt level requirements and system level assumptions in the given architecture. AGREE uses an SMT based solver for the analysis, that creates a proof of correctness for every requirement. Our approach is to extract the claims in the proof that are nothing but component requirements and assumptions that were neccessary to satsify each system level requirement. We use an existing capabiltiy of the SAT solver to query such claims - called UNSAT core. What we get from the USAT core is nothing but a satsifaction argument - the specific component requirements and assumptions - for a requirement. Each element in the satisfaction argument is called the\textbf{ Set of support }for the requirement.  Ofcourse, it is only one satisfaction argument from that proof. Any one who has used SAT based solving might be aware that there could be many such proofs. So, by getting all possible proofs for a requirement, we can get all possible Sets of support. That gives us a complete traceability for a requirement.

While extracting the traceability without exsting trace links itself is useful, the complete set of support that we got is insightful. We categorized the items in the set of support for every requirement into two: One is MUST elements that are present in all sets of support and the rest in MAY group. MUST group elements indicated that they are absolutely necessary for the requirements to satisfied. Any change to them will impact the satisfaction of the requirements. Whereas the MAY elements are optional ways to satisfy a requirement. If we just change one of the MAY elements, it the requirements will still hold. On the other hand, if we consider all the requirements's set of support, then we can analyse the change in the element not just with respect to one requirement but to the system as whole. Alternatively, we could also analyse which component has the most MUST elements to determine the criticality of that component



\section {Related work}
How trace links are established can be categorized into  (a) Manual (b) Automatic (c) Semi-automatic

I will elaborate on these.. There are too many, I need to carefully see which is very relvant to what we are saying here. For Requirements - component requirements and for MBD specifically.


\section {Background}


In this section, I will explain our compositional approach work.

AGREE tool quick explanation on how it proves

\section{Our Approach}


\subsection{Set of Support}

What is a proof  - claim -

What is UNSAT core

and what is set of support ( I will put the proof picture here )


\subsection{Traceability}

Condense the approach to derive it. Need to include information from the AGREE side back to AGREE also in addition to things in JKIND.


( I will put the picture of GPCA traceability here)


Concentrate on the overall implementation and abstract away jKind details. Cite the other paper.


\subsection{Complete Traceability}

 This is the section, I am worried about, since we havent implemented it. But we can write the algorithm here and explain it.




 \emph{$ i_s > 10 \Rightarrow o_s > 20 $}, that is expressed in terms of its  input $i_s$ and output $o_s$, as shown in Figure~\ref{fig:toy}. Say, the engineers decide to compose two components A and B so that they together satisfy the system requirements. Each of these components, have their own inputs, outputs and a set of requirements they satisfy ($R_{a1}, R_{a2}, R_{b1}$ and $R_{b2}$). To enable composition, the engineers connect their inputs and outputs and check if their individual requirements are sufficient to satisfy the system level requirement that architecture (the way the components and system are connected). This is a typical activity that happens in most complex system developments. In model based developments, architectural models of the system and its components are developed and analyzed.



As one could see, the $R_s$ is satisfied by the components, since $R_a1$ and $R_b1$ together imply $R_s$. Instead of just recording a link between  $R_s$ to $R_a1 and R_b1$ separately, satisfaction arguments, originally proposed by Zave and Jackson, suggested to record it as ($R_a1 and R_b1$ satisfy $R_s$). This way of capturing the association is enormously useful since it demonstrates how the behaviors of the components satisfy the requirements. But, what was not clarified by Zave and Jackson is whether one should record just one or all possible satisfactions. In this example, an alternate satisfaction argument is $R_{a2} and R_{b2}$ satisfy $R_s$. If one intends to perform impact analysis to change $R_{a1}$ then the fact that the satisfaction of $R_S$ is not affected at all is not visible unless the alternate traceability was recorded. We believe that this is a crucial aspect that is not well discussed in traceability. Although, \emph{Rich traceability} and Goal decompositions introduced the notion of AND and OR composition of component requirements/goals in the way they contribute to satisfying the system requirement/goal, but they did so in the context of recording alternate design choices but not in the case of actually tracing to alternate means of satisfying the requirement.




\section{Background and Related work}
\anitha{I need to see if I need related work here or put it later}
Model based development is a emerging as a widely accepted technique to developing systems, especially in the safety critical systems domain. In MBD, models are the central artifacts. A variety of MBD tools are used to automatically and efficiently analyse the model with respect to their requirements, expressed using some form of formal logic. This has been a very successful development technique since it helps identify errors early in the development cycle and help improve the quality of the artifacts while reducing code. However, with the increase in the size and complexity of the models, it is not easy to understand how and where the requirements are implemented in the models. However, if established, such as an association, in other words traceability, helps effectively perform :
\begin{itemize}
  \item \textbf{Impact analysis} - To determine which parts of the model implements the requirement, how a change in requirement affects the model and check if all requirements are implemented by its components.
  \item \textbf{Dependency analysis} -  To identify which requirement had given rise to specific parts of the model, identify which requirement will be affected if the component is changed, assess the coverage of the requirements over the model, determine which parts of the components do not trace to any requirement.
\end{itemize}

Unfortunately, one of the major challenges in MDB is precisely establishing traceability between the system requirements and the parts of the model that implements it. The main reason behind the difficulty is the inherent complexity of the models that makes the process of manually establishing trace links exhausting. However, there has been some research efforts to automatically establish the traceability between requirements and model elements. Some of them include :
\begin{itemize}
  \item Event-Based Traceability (EBT) (Cleland-Huang et al) is a method that automatically establishes trace links and maintains it, between requirements to models, using event-based mechanisms.
  \item Egyed and Grunbacher enhance the basis from which they derive links by following the approach of dynamic program analysis: They record program execution traces (these are—as described earlier—dynamic program behavior logs) from the execution of test-cases. They combine the results with a set of requirements-to-code traceability links, which have to be established manually beforehand and infer traceability links between requirements.
  \item Wenzel et al. describe a tool which is able to compare the version history of models, to derive traceability links, and consequently, to reconstruct the evolution steps of single model elements.
  \item alloy work here or at the end?
\end{itemize}
\anitha{I have a survey paper, I will fill in relevant work here}

While the above techniques establish trace links, without a rationale for why and how they were established they are often inadequate to perform useful analysis in practice~\cite{Trace links explained: An automated approach for generating rationales}. In that context, \textbf{\emph{Satisfaction Arguments}} offer a meaningful way to establish traceability between functional requirements and its realization. Originally proposed by Zave and Jackson, satisfaction argument is an approach to demonstrate how the behaviors of the system along with the assumptions made about its environment satisfy the functional requirements. These arguments have been enormously useful in establishing traceability between the requirements and parts of the system that implements them. Unlike traditional traceability that just links two artifacts, satisfaction argument is aimed to provide an explanation to why the traceability was established.

\textbf{In previous work we} demonstrated an approach to system construction in which compositional proofs are used to to establish satisfaction arguments. Given an architectural model of the system (decomposition of system into components) in which each component (including the system) is endowed with its own behavioral requirements and assumptions it makes about its environment, our approach compositionally verifies if system level requirements as a logical consequence of the component level requirements. To demostrate our approach, we used a tool called AGREE \cite{NFM2012:CoGaMiWhLaLu} -- a compositional reasoning framework based on assume-guarantee reasoning ~\cite{McMillan99:circ}, that provides an appropriate mechanism for formally capturing the requirements, and assumptions to verify system requirements. \anitha{Do I need more explanation here?}.

AGREE uses JKind, a bounded model checker, to verify the requirements of the system. Model checking is a technique that allows reasoning whether the requirements expressed using temporal logic notion are satisfied by the model of the system. Given an architectural model and the requirements of each component, AGREE and JKind transforms the model into a logical formula that defines how the system can evolve from one time instant to the next time instant, starting from an initial state. This transformed logical formula provides the structure for the model checker to prove the requirements of the system using the principle of mathematical induction. If the model checker finds a violation of the requirement, it reports a counter example, if not it declares that the requirement is valid.

\anitha{i took few lines from the other paper}
While the compositional verification was very useful in proving system level requirements, in the event that requirement is proved, it is not always clear what level of assurance should be invested in the result.  Given that these kinds of analyses are typically performed for safety critical system, this can lead to overconfidence in the behavior of the fielded system. It is well known that issues such as vacuity~\cite{Kupferman03:Vacuity} can cause verification to succeed despite errors in a requirements or in the model. Even for non-vacuous requirements, it is possible to over-constrain the {\em environment} of the model such that the implementation will not work in the actual operating environment. Hence, to gain confidence over the verification we pursed an approach that would provide us with an evidence of the successful verification. An evidence in this context is nothing but an explanation about which parts of the model (the component requirements and system assumptions) the model checker used to prove the system level requirement. In the next section, we briefly explain this approach (the details are available elsewhere) and our approach to leverage this information to establish traceability.


\emph{\textbf{Traceability}} in system development in its simple sense is the association that is established between artifacts for a specific purpose. The association between the artifacts is commonly called the trace link~\cite{gotel2012traceability}. Associating the requirements of the system with other related artifacts such as design, code, test case etc., is called \emph{Requirements Traceability}. An important subset of requirements traceability involves tracing the {\em functional} requirements, that describe expected behaviors of the system to be constructed, to the actual parts of the system that realizes those requirements. Such an association is useful for various purposes such as analysing how requirements are realized in the system, identify how a change in requirements affects the existing realization, determine which parts of the system does not realize any requirement, check if the requirements sufficiently describe all aspects of the system behaviour and much more.

